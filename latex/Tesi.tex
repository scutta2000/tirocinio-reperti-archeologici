%
% Tesi D.S.I. - modello preso da
% Stanford University PhD thesis style -- modifications to the report style
%
%%%%%%%%%%%%%%%%%%%%%%%%%%%%%%%%%%%%%%%%%%%%%%%%%%%%%%%%%%%%%%%%%%%%%%%%%%%
%                                                                         %
%			TESI DOTTORATO                                                   %
%			______________                                                   %
%                                                                         %
%			AUTORE: Elena Pagani                                             %
%                                                                         %
%			Ultima revisione: 7.X.1998                                       %
%           correzioni atrent                                             %
%%%%%%%%%%%%%%%%%%%%%%%%%%%%%%%%%%%%%%%%%%%%%%%%%%%%%%%%%%%%%%%%%%%%%%%%%%%
%
%
\documentclass[a4paper,12pt]{report}
%    \renewcommand{\baselinestretch}{1.6}      % interline spacing
%
% \includeonly{}
%
%			PREAMBOLO
%
\usepackage[a4paper]{geometry}
\usepackage{amssymb,amsmath,amsthm}
\usepackage{graphicx}
\usepackage{url}
\usepackage{hyperref}
\usepackage{epsfig}
\usepackage[italian]{babel}
\usepackage{setspace}
\usepackage{tesi}

% per le accentate
\usepackage[utf8]{inputenc}
%
\newtheorem{myteor}{Teorema}[section]
%
\newenvironment{teor}{\begin{myteor}\sl}{\end{myteor}}
%
%
%			TITOLO
%
\begin{document}
\title{Tecniche di machine learning per la classificazione di reperti archeologici}
\author{Pietro Scuttari}
\dept{Corso di Laurea in informatica} 
\anno{2020-2021}
\matricola{922822}
\relatore{Prof.ssa Anna Maria Zanaboni}
\correlatore{Prof. Dario Malchiodi}
%
%        \submitdate{month year in which submitted to GPO}
%		- date LaTeX'd if omitted
%	\copyrightyear{year degree conferred (next year if submitted in Dec.)}
%		- year LaTeX'd (or next year, in December) if omitted
%	\copyrighttrue or \copyrightfalse
%		- produce or don't produce a copyright page (false by default)
%	\figurespagetrue or \figurespagefalse
%		- produce or don't produce a List of Figures page
%		  (false by default)
%	\tablespagetrue or \tablespagefalse
%		- produce or don't produce a List of Tables page
%		  (false by default)
% 
%			DEDICA
%
\beforepreface
\prefacesection{}
		{\hfill \Large {\sl dedicato a \dots}}
% 
%			PREFAZIONE
%
\prefacesection{Prefazione}
hkjafgyruet.
%
%
%			ORGANIZZAZIONE
\section{Organizzazione della tesi}
\label{organizzazione}
La tesi \`e organizzata come segue:
\begin{itemize}
	\item Nel capitolo 1 viene introdotto il progetto indicando lo scopo del lavoro e 
introducendo i concetti principali
	\item Nel capitolo 2 

\end{itemize}
%
%			RINGRAZIAMENTI
%
\prefacesection{Ringraziamenti}
asdjhgftry.
\afterpreface
% 
% 
%			CAPITOLO 1: Introduzione
\chapter{Introduzione}
\label{cap1}
\section{Descrizione}
Il progetto consiste nel classificare un database di analisi di composizioni eseguite
su dei reperti archeologici. La classificazione è stata eseguita in base all'origine
geografica distinguendo i reperti originari di Tarquinia, luogo dove sono stati 
ritrovati, da quelli di origine diversa. I classificatori sono per lo più 
supervisionati e allenati su una porzione dei reperti di cui conoscevamo in partenza
l'origine.

\section{Cos'è il machine learning} %Vanno aggiunte le fonti
Machine learning è un nome che include una varietà di algoritmi che, al contrario
di algoritmi tradizionali, non specificano passo per passo come risolvere un
certo problema ma migliorano gradualmente imparando da dati fino a risolvere 
correttamente il problema. \\
Questo approccio ha origini storiche già negli anni cinquanta, già Alan Turing 
propone un'ipotetica macchina in grado di imparare e diventare intelligente. Negli 
ultimi anni abbiamo visto realizzare il vero potenziale di questo approccio:
con l'aumento esponenziale della potenza dei calcolatori e l'enorme quantità di dati
oggi disponibili il machine learning è applicato a sempre più problemi, dai veicoli 
autonomi, agli algoritmi per la selezione della pubblicità a microscopi in grado di 
identificare cellule cancerogene.

\section{Cosa sono i problemi di classificazioni}
% I problemi di classificazione sono un insieme di problemi in cui dobbiamo dividere 
% un insieme di oggetti in una serie di categorie.

La classificazione è un sottoinsieme del machine learning, l'obbiettivo è costruire un 
modello in grado di mappare un oggetto in ingresso con una categoria. I classificatori 
si distinguono in due macro categorie quelli a apprendimento supervisionato e quelli a
apprendimento non supervisionato: i primi imparano a classificare correttamente a 
partire da un database di dati etichettati ovvero dove è specificata la categoria 
corretta, nel secondo caso i dati di addestramento non hanno memorizzato le etichette.
Evidentemente il secondo caso è più complesso sia per l'addestramento del modello sia 
per misurare la sua correttezza. In questo progetto sono stati utilizzati principalmente
algoritmi di apprendimento supervisionato.


\chapter{Principali modelli per la classificazione}
\section{Network neurali}
\section{K-nearest neighbors}
\section{Macchine a vettori di supporto}
\section{Alberi di decisione}
\section{K-means}

% 
% 
%			CAPITOLO 2: Il problema affrontato
\chapter{Il problema affrontato}
\label{cap2}
\section{Descrizione dei dati}
\section{Ambiente software}
\section{Schema delle prove}
\subsection{Repeted hold out}
\subsection{Convalida incrociata}
\subsection{Griglia di ricerca}


% 
% 
%			CAPITOLO 3: Risultati
\chapter{Risultati}
\label{cap3}
\section{Valutazione combinata}

% 
% 
%			CAPITOLO 4: Conclusioni
\chapter{Conclusioni}
\label{cap4}

%
%			BIBLIOGRAFIA
%
\begin{thebibliography}{00}
%
\bibitem{gotti91}
M. Gotti, I linguaggi specialistici, Firenze, La Nuova Italia, 1991.
%
\bibitem{kriesel07}
D. Kriesel, A brief introduction to neural networks, available at http://www.dkriesel.com, 2007. %Seguendo "how to cite" sul sito
%

\end{thebibliography}
% 
\end{document}


 
